\begin{abstract}\noindent
Questa relazione espone i risultati inerenti all'esercitazione di rilievo dendrometico, svolta nell'azienda agricola sperimentale di Vallevecchia, Caorle (Ve). L'area di studio è stata divisa in due particelle, est ed ovest.\\
Al fine di ricavare i principali parametri dimensionali (densità di popolamento, diametro medio,...), sono stati utilizzate tre diverse metodologie di rilievo: il cavallettamento totale, il metodo delle aree di saggio circolari ed il metodo relascopico.\\
Confrontando i risultati, si può notare come i metodi di rilevamento campionario sovrastimino i parametri di densità di popolamento e di area basimetrica, rispetto a quelli del censimento. Inoltre, l'analisi dei risultati permette di capire come i due popolamenti siano completamente differenti, pur essendo a poca distanza tra di loro.
\end{abstract}

\tableofcontents