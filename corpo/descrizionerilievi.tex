\section{Descrizione dei rilievi}
La zona coperta da alberi, e quindi d'interesse per il rilievo, è stata divisa in due particelle, la ovest e la est. A loro volta, queste aree sono state divise in due sottoparticelle. Si avranno quindi due sottoparticelle ovest (da  1.14ha e 1.10ha), e due sottoparticelle est (da 1.17ha e 1.55ha).
Al fine di ricavare i parametri dendrometrici necessari per comprendere il popolamento, sono stati utilizzati tre diverse modalità di rilievo dei diametri, con misure campionarie delle altezze.\\
Le rilevazioni della particella ovest sono state svolte il 24 maggio 2023, mentre la particella est è stata analizzata il 26 maggio 2023.\\
In tutte e tre le tipologie di misurazione, sono state presi in considerazione solamente gli alberi vivi, con soglia di cavallettamento di 17,5 cm.\\

\subsection{Cavallettamento totale}
Consiste nella delimitazione di un'area (in questo caso una sottoparticella) e la successiva rilevazione dei diametri di ciascun albero presente all'interno di essa. Dovendo ricavare la misura di molti alberi, il cavallettamento totale risulta un'operazione lunga, spesso poco efficiente. In questa operazione di rilevazione possono presentarsi degli errori dovuti all'operatore ed agli strumenti (per esempio se tarati male).\\
In questa esperienza, i gruppi si sono disposti in parallelo e hanno misurato ogni singolo albero, coprendo quindi tutta la zona precedentemente delimitata.\\
Per svolgere questo tipo di rilevazione, ogni squadra ha utilizzato un cavalletto dendrometrico (oppure una cordella metrica) un ipsometro meccanico e dei gessi (per evitare di misurare più volte lo stesso albero).
\subsection{Aree di saggio con raggio fisso}
Consiste nella delimitazione di un'area circolare (in questo caso di 10m di raggio), attorno a un albero prestabilito. Successivamente si compie il cavallettamento diametrico di tutti gli alberi all'interno di quest'area, e di misure campionarie di altezze. In questa esperienza, ogni squadra ha compiuto i cavallettamenti in due aree di saggio diverse, ma all'interno della stessa sottoparticella.\\
Inevitabilmente, agli errori causati dall'uomo e dagli strumenti, si aggiungono gli errori dovuti all'inferenza statistica, ovvero al relazionare questi valori alla superficie di un ettaro. \\
Per svolgere queste misurazioni, ogni squadra ha utilizzato un cavalletto, un dendrometro digitale (Vertex o Trupulse), una cordella metrica e dei gessi.
\subsection{Aree relascopiche diametriche}
Consiste nel posizionarsi a piacere in un punto dell'area d'interesse, e utilizzando il relascopio, contare tutti gli alberi che rientrano all'interno della banda prestabilita.\\
Anche in questo caso, ogni squadra ha compiuto rilevazioni in due punti diversi, ma all'interno della stessa sottoparticella; inoltre, sono state fatte misure campionarie di altezze.\\
Come per le misurazioni di aree di saggio con raggio fisso, questa metodologia porta a errori, causati dall'operatore, dallo strumento e dal calcolo statistico (per rapportare i risultati all'ettaro).\\
Per questa serie di misure, ogni squadra ha utilizzato un relascopio, un cavalletto, un ipsometro, una cordella metrica e dei gessi.