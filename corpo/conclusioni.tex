\section{Conclusioni}
Al termine dello studio delle due particelle, utilizzando metodi diversi di rilevamento, è possibile compiere alcune osservazioni.\\
I risultati hanno chiarito e confermato le composizioni e le proprietà delle due aree di studio.\\ 
Secondo il cavallettamento totale, la particella ovest è composta da un bosco, prevalentemente di pino domestico, con: portamento biplano, diametro medio di 32 cm, altezza media (di p. pinea) di 15,5 m area basimetrica totale di 84,8 $\frac{m^3}{ha}$ e densità di 467 individui a ettaro. Sempre secondo il cavallettamento totale, la particella est invece, è composta da una pineta monoplana, di soli pini domestici, con: diametro medio di 28 cm, altezza media 13,6 m, area basimetrica di 44 $\frac{m^3}{ha}$ e una densità di 709 alberi a ettaro.\\ 
Risulta quindi, che la particella ovest ha un diametro medio, un'area basimetrica e un'altezza (almeno per il pino domestico), maggiore rispetto a quella est, che però ha una densità di individui a ettaro superiore.\\
Comparando i tre metodi, separatamente per la particella est e ovest, si può notare come le aree di saggio circolari e quelle relascopiche portino a valori di numerosità e area basimetrica a ettaro maggiori rispetto a quelle ricavate mediante cavallettamento totale.\\
Queste notevoli differenze potrebbero essere causate da tre motivazioni: errori derivanti l'inferenza, derivanti gli errori sistematici dell'operatore e dello strumento, e dalla scelta soggettiva delle aree di saggio. Di fatto, la presenza di rovi induce l'operatore a evitare il campionamento di quella zona, portandolo invece verso un'area libera da spine, ma magari con una presenza maggiore di alberi. Inoltre, al fine di ridurre l'errore standard, nei casi di campionamenti, si cerca di rilevare le zone di bosco dove c'è una numerosità di campioni maggiore. Il campionamento in aree al limite del bosco porta a rilevamenti di zone con densità minori di alberi.\\
Per quanto riguarda il calcolo dei volumi, le due formule hanno portato a differenze di valori, all'interno delle stesse particelle, notevoli. Di fatto, le formule delle tavole Tabacchi sovrastimano la biomassa a ettaro rispetto alle corrispettive formule Ravenna.\\ \\
Per ridurre gli errori, compiuti durante l'esercitazione, si potrebbe adottare l'utilizzo di campionamenti non soggettivi, come per esempio quello sistematico oppure quello casuale, pur tenendo in considerazione la possibilità che ci possano essere distorsioni o aree non sufficientemente rappresentate.\\
Rimane il fatto che, il cavallettamento totale elimina le fonti di errori causate dalle trasformazioni statistiche e dall'errata scelta di aree di saggio. Seppur sia un metodo efficace, risulta poco efficiente, necessitando di molto tempo per le misurazioni e per gli spostamenti.\\
Un'altra possibilità per ridurre gli errori (e velocizzare le operazioni in bosco) è quella di utilizzare strumenti ottici di misura, al posto di quelli analogici; come per esempio, l'utilizzo del Vertex al posto della cordella metrica e dell'ipsometro di Blume-Leiss. Di fatto, la minore sensibilità alle oscillazioni dell'operatore, la non necessità di correzione della pendenza e della misura della distanza, permette di migliorare l'efficacia e l'efficienza delle misurazioni, migliorando la bontà dei risultati finali. 